\documentclass[oneside]{ausarbeitung}
\bibliography{latexlit}

\usepackage{tikz}
\usepackage{listings}
\lstset{numbers=left, numberstyle=\tiny, numbersep=5pt}
\lstset{language=C}
\lstset{ % ... whatever was already there ...
        literate=% ... any other literates already there ...
                 {!}{!}1
                 {?}{?}1
                 {:}{:}1
}

\usepackage{xcolor}

\colorlet{punct}{red!60!black}
\definecolor{background}{HTML}{EEEEEE}
\definecolor{delim}{RGB}{20,105,176}
\colorlet{numb}{magenta!60!black}

\lstdefinelanguage{json}{
    basicstyle=\normalfont\ttfamily,
    numbers=left,
    numberstyle=\scriptsize,
    stepnumber=1,
    numbersep=8pt,
    showstringspaces=false,
    breaklines=true,
    literate=
     *{0}{{{\color{numb}0}}}{1}
      {1}{{{\color{numb}1}}}{1}
      {2}{{{\color{numb}2}}}{1}
      {3}{{{\color{numb}3}}}{1}
      {4}{{{\color{numb}4}}}{1}
      {5}{{{\color{numb}5}}}{1}
      {6}{{{\color{numb}6}}}{1}
      {7}{{{\color{numb}7}}}{1}
      {8}{{{\color{numb}8}}}{1}
      {9}{{{\color{numb}9}}}{1}
      {:}{{{\color{punct}{:}}}}{1}
      {,}{{{\color{punct}{,}}}}{1}
      {\{}{{{\color{delim}{\{}}}}{1}
      {\}}{{{\color{delim}{\}}}}}{1}
      {[}{{{\color{delim}{[}}}}{1}
      {]}{{{\color{delim}{]}}}}{1},
}

% ----------------------------------------------------------------------

\begin{document}

\selectlanguage{ngerman}

%--- Art der Arbeit
% Erlaubte Werte:
% Praxissemesterbericht, Projektbericht, Bachelorarbeit oder                                % Masterarbeit
\doctype{Bachelorarbeit}

%--- Studiengang:
\depname{Medieninformatik}

\title{WebGPU}

\author{Laurin Agostini}
\matrikelnr{60526}

\examinerA{Prof.~Dr.~Winfried~Bantel}
\date{XX. Juni 2020}

%--- Titelseite Anzeigen
\maketitle
\cleardoublepage

%---
\pagenumbering{roman}
\setcounter{page}{1}


%--- Eidesstattliche Erklärung anzeigen
\makeaffirmation
\cleardoublepage

%---
\chapter*{Kurzfassung}
\addcontentsline{toc}{chapter}{Kurzfassung}
In dieser Bachelorarbeit geht es um die neuartige Grafik-API WebGPU, die einen Nachfolger zu WebGL darstellt. Mit WebGPU ist es möglich, detaillierte 3D-Szenen, aufwendige Simulationen und XXX in Echtzeit direkt im Webbrowser zu berechnen und darzustellen. 

%-----------------------------------------------------------------------
\cleardoublepage
\addcontentsline{toc}{chapter}{Inhaltsverzeichnis}
\tableofcontents

%---
\addcontentsline{toc}{chapter}{Abbildungsverzeichnis}
\listoffigures

%---
\addcontentsline{toc}{chapter}{Tabellenverzeichnis}
\listoftables

%---
\chapter*{Abkürzungsverzeichnis}
\addcontentsline{toc}{chapter}{Abkürzungsverzeichnis}
\begin{acronym}[JSON]  % Längstes Kürzel in der nachfolgenden
                       % Liste um die Breite der Spalte für die
                       % Abkürzungen zu bestimmen.

%% Eintrag: \acro{Referenzname}[Kürzel]{Langform}
%% Im Text wird die Abkürzung dann mit \ac{Referenzname} benutzt.
\acro{zb}[z.B.]{zum Beispiel}
\acro{dh}[d.h.]{das heißt}
\acro{json}[JSON]{JavaScript Object Notation}
\end{acronym}
%---
\cleardoublepage
\pagenumbering{arabic}
\setcounter{page}{1}

% ----------------------------------------------------------------------
\chapter{Einleitung}
\label{cha:einleitung}

\section{Motivation}
\label{sec:motivation}

\section{Problemstellung und -abgrenzung}
\label{sec:problemstellung}

\section{Ziel der Arbeit}
\label{sec:ziel}


\section{Vorgehen}
\label{sec:vorgehen}


% ---
\chapter{Grundlagen}
\label{cha:grundlagen}
%---
\chapter{Problemanalyse}
\label{cha:problemanalyse}


%---
\chapter{Implementierung}
\label{cha:implementierung}



%---
\chapter{Evaluierung}
\label{cha:evaluation}


%---
\chapter{Zusammenfassung und Ausblick}
\label{cha:zusammenfassung}
\section{Erreichte Ergebnisse}
\label{sec:ergebnisse}


\section{Ausblick}
\label{sec:ausblick}

\subsection{Erweiterbarkeit der Ergebnisse}
\label{sub:erweiterbarkeit}

\subsection{Übertragbarkeit der Ergebnisse}
\label{sub:uebertragbarkeit}

%-----------------------------------------------------------------------
\appendix

%---
\printbibliography

%---
%\chapter{Anhang A}

%---
%\chapter{Anhang B}


\end{document}